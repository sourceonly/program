\chapter{How to boot linux live-cd directory from Grub2}
This document provide the way to chainload an iso-file (Live CD iso) from either floppy or hardisk, without unpacking it. I used the Centos-Live CD

\section{Preparation}

Grub bootable USB-stick:
\begin{lstlisting}
  #fdisk -l /dev/sdb
  Disk /dev/sdb: 3.8 GiB, 4023385600 bytes, 7858175 sectors
  Units: sectors of 1 * 512 = 512 bytes
  Sector size (logical/physical): 512 bytes / 512 bytes
  I/O size (minimum/optimal): 512 bytes / 512 bytes
  Disklabel type: gpt
  Disk identifier: 3C5D7777-4CAF-4C10-A1F9-BAAA9AB3CC61

  Device       Start     End Sectors  Size Type
  /dev/sdb1     2048  264191  262144  128M BIOS boot
  /dev/sdb2   264192 1312767 1048576  512M EFI System
  /dev/sdb3  1312768 7858141 6545374  3.1G Microsoft basic data
\end{lstlisting}

with this USB device, we can boot either in \emph{grub lecacy mode} or \emph{grub efi mode}, and have \emph{Centos Live ISO} in {\ttfamily /dev/sdb3}

\begin{lstlisting}
  # mount /dev/sdb3 /mnt
  # ls -ltr /mnt
  total 2263064
  -rw-r--r--  1 source source 1793064960 Dec  2 22:26 CentOS-7-x86_64-LiveKDE-1708.iso
  drwxr-xr-x  2 source source       4096 Dec  3 10:37 System Volume Information
  drwxr-xr-x  3 source source       4096 Dec  3 23:36 centos
  drwxr-xr-x  2 source source       4096 Dec  3 23:37 squash
  drwxr-xr-x  2 source source       4096 Dec  3 23:38 ext3
  drwxr-xr-x 17 source source       4096 Dec  3 23:40 root
  -rw-r--r--  1 source source  524288000 Dec  4 03:23 cowfile.out
  -rw-r--r--  1 source source        488 Dec  4 03:27 get_root_ready
\end{lstlisting}


\section{Boot with Grub ...}
During start up, press enter to select from boot device, here either lecacy mode or efi mode would work. As we don't have any \ttfamily{grub.cfg} file to customize menu, we only have \emph{grub command line interface}, if you have a customized grub menu, as it said, hit 'c' to enter \emph{grub command line interface}

\begin{lstlisting}
  grub> 
\end{lstlisting}

First of all, we need to make the iso file as a loop device in grub2, feel free to use \emph{Tab} to give you some hints. 
\begin{lstlisting}
  grub> loopback loop0 (hd0,gpt3)/CentOS-7-x86_64-LiveKDE-1708.iso
\end{lstlisting}
after this, you would have a device called {\ttfamily loop0}, continue boot from command line, we don't need \emph{isolinux.bin} any more as grub2 takes over the role of it.
\begin{lstlisting}
  set root=(loop0)
  linux /isolinux/vmlinuz0
  initrd /isolinux/initrd0.img
  boot
\end{lstlisting}
the boot would definately failed as \emph{initrd0.img} would actually not find it's real root. Later we can see it would fail at {\ttfamily switch-root}.

now we have a minial linux enviroment within \emph{initrd0.img}, and we are trying to tell the system where the rootfs actually lies in.

\section{Get real root ready}

The root file system lies in \emph{ISO file}. First of all we need to find where it is, feel free to change the mount point as you wish here. 

\begin{lstlisting}
  # mkdir -p /target/usb
  # mount /dev/sdb3 /target/usb
  # mkdir -p /target/iso
  # mount /target/usb/CentOS-7-x86_64-LiveKDE-1708.iso /target/iso/
  # tree /target/iso
  /target/iso/
  ├── isolinux
  │   ├── boot.cat
  │   ├── initrd0.img
  │   ├── isolinux.bin
  │   ├── isolinux.cfg
  │   ├── vesamenu.c32
  │   └── vmlinuz0
  └── LiveOS
      ├── osmin.img
      └── squashfs.img

\end{lstlisting}

the root file system is under {\ttfamily LiveOS/squashfs.img}, again we need to mount it manually,

\begin{lstlisting}
  # mkdir -p /target/squashfs
  # mount /target/iso/LiveOS/squashfs.img /target/squashfs
  # tree /target/squashfs
  /target/squashfs/
  └── LiveOS
   └── ext3fs.img
\end{lstlisting}

here, unfortunately, we can't mount ext3fs.img directly, as it's a readonly file system. We can't write anything into it. Force switch-root to it would lead to a lot of errors.

We need to make a snapshot of it, along with some writeable blocks. we choose \emph{loop5} and \emph{loop6} as their name. the size of output.file is up to you to choose.
\begin{lstlisting}
  # losetup /dev/loop5 /target/squashfs/LiveOS/ext3fs.img
  # dd if=/dev/zero of=/dev/shm/output.file bs=8k count=20000 
  # losetup /dev/loop6 /dev/shm/output.file
\end{lstlisting}

after this, we need to make a snapshot with \emph{loop5} and \emph{loop6}, where \emph{loop6} stores actually file changes.

\begin{lstlisting}
  # echo "@\$@(blockdev --getsize /dev/loop5) snapshot \
  /dev/loop5 /dev/loop6 p 64" | dmsetup create rootfs
\end{lstlisting}

now we have a snapshot of it, it's out readable and writable device , mount it somewhere
\begin{lstlisting}
  # ls /dev/dm-0
  # mkdir -p /target/rootfs
  # mount /dev/dm-0 /target/rootfs
\end{lstlisting}


  
\section{Before switch root ...}
Hold on and take a deep breath, don't be too overjoyed to switch\_root now, you would never login with the new user root, as you don't known it's passwd.
so we need to settle it before it goes bad. The easist way to do this is using chroot. 
\begin{lstlisting}
  # chroot /target/rootfs
  # passwd
  < type in your passwd for root>
  # useradd test1
  # passwd test1
  < type in your passwd for test1>
\end{lstlisting}

turn off selinux

\begin{lstlisting}
  # vi /etc/sysconfig/selinux  
  <change selinux to disabled>
  
\end{lstlisting}

if you don't want turn off selinux, trying following way.
\begin{lstlisting}
  # touch /.autorelabel
\end{lstlisting}
here we have user root/test1, and the creditial you know. After doing this, hit \emph{Ctrl-D} to exit chroot. 

\section{Finally}
Finally,  all is done. let's go back to where switch root fails. you can refer to \emph{/usr/lib/systemd/system/initrd-switch-root.service} for the execution command. 
\begin{lstlisting}
  #  /usr/bin/systemctrl --no-block --force switch-root /target/rootfs
\end{lstlisting}

enjoy! 
\cleardoublepage

