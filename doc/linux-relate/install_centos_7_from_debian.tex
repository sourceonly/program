\chapter{Install CentOS 7 from debian}
In this section we would descript how to install a new CentOS from debian
\section{Why}
because i use debian personlly, and need do some test on centos, I'm not sophisticated on \emph{Xen}, \emph{KVM} or other tech. I would need old and always work ways, reboot and then select your system.
\section{Require}
Here, we assume (that what is it on my labtop}
\begin{enumerate}
\item you use lvm, and have enough diskspace
\item you have network connection, though both {\ttfamily apt } and {\ttfamily yum} would work. 
\end{enumerate}

\bold{Debian} use {\ttfamily dpkg/apt} to manage it's packages, while CentOS use {\ttfamily rpm/yum}, luckily enough, we also have {\ttfamily rpm/yum} in debian repository

we need to install it
\begin{lstlisting}
 #  apt-get -y install yum 
\end{lstlisting}
and add a repository,  here we use \url{http://mirrors.163.com/centos}(which is also my debian repository, here the path is slightly different from centos, which is {\ttfamily /etc/yum/repos.d} rather than {\ttfamily /etc/yum.repos.d}
\begin{lstlisting}
  #  cat /etc/yum/repos.d/base.repo
  [base]
  name=os
  baseurl=http://mirrors.163.com/centos/7.4.1708/os/x86_64
  gpgcheck=0
  enable=1
\end{lstlisting}

\section{Install CentOS 7 ... }
We can get the job done really directly
\subsection{Create \emph{root fs}}
To make things simple, we use \emph{lvm} to create rootfs dev , here \emph{vg} is my \emph{lv} which have more than \emph{50g} free space
\begin{lstlisting}
  #  lvcreate -L50g -n centos vg
  #  mkfs.ext4 /dev/mapper/vg-centos
  #  mkdir -p /centos
  #  mount /dev/mapper/vg-centos /centos
\end{lstlisting}

\subsection{Install system with yum ...}
Make yum happy..
\begin{lstlisting}
  # yum --installroot=/centos makecache
  # yum --installroot=/centos install yum
  # yum --nogpgcheck --installroot=/centos install @core kernel
\end{lstlisting}

if you are trying in LiveCD, you'd better install grub, here we already have grub2 on my disk along with debian
\begin{lstlisting}
  # yum --installroot/centos install grub2
\end{lstlisting}
here we just run \emph{update-grub}, it would auto detect centos presents
\begin{lstlisting}
  # update-grub
  Found background image: /usr/share/images/desktop-base/desktop-grub.png
  Found linux image: /boot/vmlinuz-4.14.0-1-amd64
  Found initrd image: /boot/initrd.img-4.14.0-1-amd64
  Found linux image: /boot/vmlinuz-3.16.0-4-amd64
  Found initrd image: /boot/initrd.img-3.16.0-4-amd64
  Found CentOS Linux 7 (Core) on /dev/mapper/vg-centos
  Adding boot menu entry for EFI firmware configuration
\end{lstlisting}
\subsection{configuation}
It's important! Otherwise you can't login to you system.
{\ttfamily chroot} would help use do this other than edit passwd file
\begin{lstlisting}
  chroot /centos
\end{lstlisting}
turn off selinux, edit /etc/sysconfig/selinux
\begin{lstlisting}
  selinux=disabled
\end{lstlisting}

change passwd for root
\begin{lstlisting}
  passwd
  
\end{lstlisting}









