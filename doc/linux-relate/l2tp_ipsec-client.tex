\chapter{Connect l2tp-ipsec vpn with Debian}

Here is only the memo of the client configuration


\section{Ipsec}
We have several choice of it, a typical work one is \ttfamily{libreswan}, with it's executable \ttfamily{pluto}

\begin{lstlisting}
  cat /etc/ipsec.d/myvpn.conf
  conn	myvpn
	authby=secret
	#pfs=no
	#rekey=no
	left=%defaultroute
	right=<ip of the server>
	rightid=<id of the server, if failed, the id should in the failed log>
	leftprotoport=17/1701
	rightprotoport=17/1701
	#dpddelay=15
	#dpdtimeout=30
	auto=start

 cat /etc/ipsec.d/myvpn.secrets
        % ANY :PSK "<you-pre-shared-key>"
\end{lstlisting}

\section{xl2tpd configure}

it should at least contain the following
\begin{lstlisting}
  cat /etc/xl2tpd/xl2tpd.conf
  [lac <you connect name>]
  name=<you connect name>
  lns=<server-ip-or-hostname>
  pppoptfile=/etc/ppp/peers/myppp.l2tpd
  ppp debug=yes
\end{lstlisting}

in \ttfamily{/etc/ppp/peers/myppp.l2tpd}
\begin{lstlisting}
remotename <you-vpn-name>
user <you-name>
password <you-passwd>
unit 0
nodeflate
nobsdcomp
noauth
persist
nopcomp
noaccomp
maxfail 5

\end{lstlisting}

\section{do connecting}
you should at least start up ipsec
\begin{lstlisting}
  ipsec restart
\end{lstlisting}

if you are with \ttfamily{auto=add}, you need to add the tunnel manually
\begin{lstlisting}
  ipsec auto --add <you_tunnel>
  ipsec auto --up <you_tunnel>
\end{lstlisting}

startup xl2tpd
\begin{lstlisting}
  systemctrl restart xl2tpd
\end{lstlisting}

after that you need to connect to it
\begin{lstlisting}
  # connect
  echo "c <you-vpn-name>" > /var/run/xl2tpd/l2tp-control
  # disconnect
  echo "d <you-vpn-name>" > /var/run/xl2tpd/l2tp-control   
\end{lstlisting}


\section{Configure route}

After all is done, you would get a new interface such as \emph{ppp0} or similarly, but the default gw sucks, to check VPN works find, you can ping with interfaces
\begin{lstlisting}
  ping -I ppp0 <internal-ip>
\end{lstlisting}



To make thing much happier, add ppp0 as the default route within VPN
\begin{lstlisting}
  route add -net <internal-net> mask <internal-mask> dev ppp0 
\end{lstlisting}

now you can directly ping or with any application you like
\begin{lstlisting}
  ping <internal-ip>
  ssh <internal-ip>
  ...
\end{lstlisting}








